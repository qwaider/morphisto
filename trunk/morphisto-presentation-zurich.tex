\documentclass {beamer}
\usepackage [ngerman] {babel}
\usepackage [T1] {fontenc}
\usepackage [ansinew] {inputenc}
\usepackage {beamerthemeshadow}
\usepackage{alltt}
\usetheme [compress] {Ilmenau}
%Ohne Navigation: default, boxes, Bergen, Bordilla, Madrid,AnnArbor, CambridgeUS, Pittsburgh, Rochester
%Mini Navigation: Berlin, Ilmenau, Dresden, Darmstadt, Frankfurt,Singapore, Szeged
%\usepackage {multimedia }
\usepackage{epsfig}
\usepackage{graphicx}
%\usecolortheme {dove}
\usecolortheme{beaver} 
%\setbeamercolor{palette quaternary}{fg=white}  
%\usecolortheme {albatross, beetle, crane, dove, seagull, wolverine, beaver}
\title{Morphisto - \\Service-oriented Open Source Morphology for German}
\author{Andrea Zielinski}
\date{26.06.2009}
\institute{FIZ}
\DeclareGraphicsExtensions{.eps,.eps.gz}

\begin{document}

\frame{\titlepage}

\subsection[Outline]{
}
%\frame{\tableofcontents}

\section{Morphisto - An Open Source Morphological Analyzer for German}

\section{Motivation}
\frame{
  \frametitle{Do we need another Morphological Analyzer for German?}
\begin{itemize}
  \item{Morphologcal analysis is the basis for a range of applications in TextGrid - an Infrastructure for ehumanities:}
\begin{itemize}
  \item{Lexical Lookup in Dictionaries}
  \item{Annotation of Texts}
  \item{Translation from/into other language stages of German}
\end{itemize}
  \item{No open-source morphological analyzer available for German at present}
  \item{This is particularly true for the lexicon component which is labour-intensive to build}
\end{itemize}
}

\subsection{Basic Idea}
\frame{
  \frametitle{Basic Idea}
\begin{itemize}
  \item{Use GPL-licensed (except the lexicon) SMOR morphology for German as a starting point}
  \item{Define lexical entries for the most common 30,000 German words as defined by the German 
reference word list \textit{DeReWo }}
   \item{Implement additional tools for the management of the lexical data}
   \item{Build easy-to-use web services and integrate them into the Eclipse Rich Client Platform}
\end{itemize}
}

\subsection{Short Introduction to SMOR}
\frame
{
  \frametitle{SMOR}

SMOR is ...
  \begin{itemize}
  \item a computational morphology for German
  \item FST-based (SFST toolkit, cf. Schmid 2004)
  \item licensed under the GPL (except the lexicon)
  \end{itemize}
}

\frame
{
  \frametitle{What it can do}
SMOR analyzes inflectional forms of German
  \begin{itemize}
  \item<1-> simple (or compound) lexemes
  \item<2-> derivational constructions
  \item<3-> complex word formations need not be stored in the lexicon but can be analyzed on-the-fly
  \item<4-> flat analyses:
%   \begin{verbatim}
    Bahn<NN>Hof<NN>Halle<+NN><Fem><Nom><Sg>
%   \end{verbatim}
  \end{itemize}
}

\section{Bootstrapping a Morphological Analyzer}
\subsection{Extracting morphological information from a MRD}
\frame
{
  \frametitle{Lexical Aquisition for MORPHISTO}
Resources used:
\begin{itemize}
\item DeReWo Lemma List \url{http://www.ids-mannheim.de/kl/derewo/}; 
\item Adelung(1793) - Grammatisch-kritisches W�rterbuch der Hochdeutschen Mundart\newline 
\url{http://www.zeno.org/Adelung-1793}; lexicon published by the ``Digitale Bibliothek"; early NHD dictionary;
free for the public; covering more than 65,000 entries.
\item Dictionary of Foreign Words (Deutsches Fremdw�rterbuch)
\item grammis 
\url{http://hypermedia.ids-mannheim.de/pls/public/gramwb.ansicht}
\end{itemize}
}
\frame{
\frametitle{Example Entry from Adelung(1793)}
An excerpt lemma entry from Adelung (1793) from which we aquired our data is given below:
\begin{enumerate}
\item Das Futter, des -s, plur. ut nom. sing. die Bekleidung eines K�rpers von au�en oder von innen; [..] 
\item Das Futter, des -s, plur. ut nom. sing. 1) Alles, was Menschen und Thieren zur Nahrung dienet; ohne Plural [..] 
\end{enumerate}
}

\subsection{Managing the Lexical Data}
\frame
{
  \frametitle{Lexical Database Scheme }
The acquired lexical data was joined with the DeReWo lemma list and stored in a lexical database.\newline  
Why is a lexical database helpful?
\begin{itemize} 
\item developers want to to add, modify, remove lexical data
\item finite-state based lexicon format is difficult to read and maintain
\end{itemize}
}

\frame
{
  \frametitle{Lexical Database Scheme }
  What is needed?
\begin{itemize}
 \item An exchange format that is independent of the specific finite-state platform
\item Scripts that convert lexical data to the originial SMOR lexicon format
\item Validating the lexicon against a schema
\item Enhanced user interface for lexicographic work 
\end{itemize}
}

%hier das Problematische...

\subsection{Cycle of Testing an Reengineering}

\frame
{ \frametitle{Cycle of Testing and Reengineering }
\begin{itemize} 
\item Checking the analysis of the DeReWo list (on the Adelung transducer)
\item Adding missing base stems (for about 5,000 entries)
\item Correcting false inflection classes or features (e.g., for \emph{Geister} (ghosts)) 
\item Adding word formation rules (e.g., for \emph{StudentIn/Innen} (male and/or female student(s))
\item Adding rules for the derivation of compound or derivation stems (e.g., \emph{Peters-kirche} (Peter's church))
\end{itemize}
}

\frame
{ \frametitle{Intermediate Result}
\begin{itemize} 
\item Manually creating the simplex units together with their required features is time-consuming
\item In the worst case, an intensive study of the documentation and software code is required
\item Gold standard would be beneficial %as only the delta of analysis that has been affected by rebuilding the network needs to be reanalyzed.
\item Fine-tuning for stems and affixes that were likely to produce ambiguities
\begin{itemize}
\item Include complex words that produce segmentation errors (e.g., \textit{Tee-nager} (tea rodent) instead of \textit{Teen-ager})
into the lexicon
\item Assign the tag <NoDef> or <Initial> to short or antiquated words to restrict their productivity
\end{itemize}
\end{itemize}
}

\section{Test Results and Discussion}

\begin{frame}
\frametitle{Transducer Lexicon Statistics for Adelung and Morphisto}
\begin{small}
\begin{table}[h] 
\begin{center}
\begin{tabular}{|l|c|c|}
\hline
\textbf{Lexicon} & \textbf{Adelung} & \textbf{Morphisto} \\
\hline
\hline
Basestems &	32152 &	17339 \\
\hline
-	Nouns 	& 20605 &	7833 \\
-	Proper Nouns &	2	& 1053 \\
-	Verb stems &	7426 &	4300 \\
-	Adjectives  &	4061	& 3178 \\
-	Adverbs 	& 2	& 781 \\
-	Closed Word Classes &	28 &	190 \\
\hline
Derivation Stems & 	63 & 	67 \\
\hline
Compound Stems	& 30	 & 181 \\
\hline
Prefix Stems	& 94	& 213 \\
\hline
Suffix Stems (Derivation Rules) & 	404 & 	410 \\
\hline
\end{tabular}
\caption{Frequency of morphological units in the transducer lexicons}
\end{center}
	  \end{table} 
\end{small}
\end{frame}

\frame
{
\frametitle{Evaluation on the 'Ispell Test Corpus'}
The wordform list provided by Ispell has been used for the evaluation of Morphisto. 
Approximately 225.833 words 
% 248.578 words - 22.745 DeReWo words 
of Ispell are unknown to Morphisto. 
The tests on randomly selected subsets of 100 inflected German wordforms in different frequency ranges reveal the number of correct, missing and spurious readings, e.g. the precision and recall rates.  
}

\frame
{
\frametitle{Evaluation on the 'Ispell Test Corpus'} 

\smallskip
\begin{table}[h] 
\begin{center}
	\begin{tabular}{ | p{3,2cm} |  p{1,6cm} | p{1,5cm} | p{2,0cm} |}
	\hline
\textbf{Frequency Classes} &  \textbf{Precision} & \textbf{Recall} & \textbf{F-Measure} \\
\hline
$F_{0} - F_{4}$ &  100.00 & 100.00 & 100.00 \\
\hline
$F_{5} - F_{8}$ &  99.63 & 99.63 & 99.63 \\
\hline
$F_{9} - F_{12}$ &  98.74 & 87.71 & 92.90 \\
\hline
$F_{13} - F_{16}$ & 98.25 & 93.85  & 95.39 \\
\hline
$F_{17}$ - $F_{20}$&  93.21 & 84.36 & 88.56 \\
\hline
$F_{21}$ - $F_{25}$  & 91.77 & 81.00 & 85.33 \\
\hline
\hline
Average &   96.93 & 91.09 & 93.63\\
\hline
\end{tabular}	
\caption{Test results on ispell for different frequency classes}
\end{center}
\end{table} 
\smallskip
}

\subsection{Discussion}
\frame
{\frametitle{FSTs in Computational Morphology}
Plus
\begin{itemize} 
\item A declarative framework for describing models
\item Phonological regularities are described by use of context dependent rewrite rules 
\item Search efficiency is improved via optimization algorithms (minimization, determinization, epsilon removal)
\item Integrate filtering constraints via composition operations
\end{itemize} 
Minus
\begin{itemize} 
\item Filter rules generate a huge network
\item Search space increases, and huge memory is required
\end{itemize}
}

\frame
{  \frametitle{Remaining Issues}  
\begin{itemize}
  \item<1-> Different Segmentations \\
   West-europa vs. Weste-ur-opa 
  \item<2-> Ambiguous Constituents \\
   der/die Chile-Kiefer
     \item<3-> Different Levels of Decomposition \\
     F\"uller vs. F\"ull-er
     \end{itemize}

\uncover<4>{Morphisto produces about 5 analyses on average!\\
  But: Humans often conceive only one reading for seemingly ambiguous words}
}

\section{Improving the Morphological Analyzer}

\frame
{  \frametitle{Dealing with Ambiguous Analysis}
Idee:\\
Define a Language Model (LM) that captures the likelihood of a morpheme sequence to filter out unlikely analyses \\
Motivation
 \begin{itemize}
\item Task is similar to PoS Tagging 
\item Hidden Markov Models (HMMs) are used a lot in unsupervised morphology learning
\item Rank all hypothesis according to the Language Model \\
in a postprocessing step
 %\includegraphics[width=5cm] {C:/LatexGraphs/Pipeline2.jpg}
 %\item Adjust lexical probabilities from DeReWo frequency statistics 
\end{itemize}
}
%This way, the analysis with the fewest constituents is not automatically favoured. 
%Note: There are restrictions based on the part of speech for derivations. For example nominal case suffixes 
%can follow only nominal stems. These are already integrated declaratively.
%But In German we have a free compounding of morphemes. This is handled by the follwing rule:
 
 
 \frame
{  \frametitle{A Language Model for Morpheme Sequences}
%HMMs are a specific transducer models. In speech they are used to map states to phonemes
% P(W) = language model (likelihood of a morpheme sequence)
% P(Y|W) = acoustic model (likelihood of observed acoustic signal given word sequence)
- PoS tag of a morpheme depends on its context (conditional word class probabilities and lexical probabilities)\\
  \begin{itemize} 
  \item \textit{parken$<$V$>$Verbot$<$+NN$>$}  \\
  \item \textit{Park$<$NN$>$Verbot$<$+NN$>$}  \\
  \end{itemize}
- Learn different PoS tag sequences from a manually annotated corpus of correct analyses (10,000 words)\\
 \begin{itemize}
\item  bigram prob.: $<$V$>$<$N$>: 0.6, lexical prob. (park$|$V) = 0.55
\item  bigram prob.: $<$N$>$<$N$>: 0.4, lexical prob. (park$|$N) = 0.45
\end{itemize}
- Disambiguate morphemes in the context of the whole word\\
  \begin{itemize}
  \item Word with morphemes \begin{math} W = m_{1} ... m_{n} \end{math}
  \item Tag sequence \begin{math} T = t_{1} ... t_{n} \end{math}
    \end{itemize}
%estimate Language Model  
 }

\frame
{  \frametitle{Issues}
\begin{itemize} 
\item Training the Language Model
\item Finding the path with the highest likelihood in the integrated morphological network
\item Smoothing
\end{itemize}
For implementation details cf. Wittl, Tilman (2009): German Morphological Disambiguator based on statistical data. In: Proceedings of TaCoS 2009.
}

\frame
{  \frametitle{Future Work}

Transfomation into a Weighted Transducer (WFST) \\
e.g. HFST, a new open-source interface to existing transducer frameworks with many add-ons  \\
\begin{itemize} 
\item SMOR grammar should be reimplemented to profit from this
\item Problems: WFST will yield even more states
\item Which weight function? Probabilities, log probabilities, costs
\item Adjustement of weights by computing lexical probabilities from DeReWo and/or use of different measures (MI, C-Value, etc.) for word class probabilities
 \end{itemize}
 %\includegraphics[width=5cm] {C:/LatexGraphs/Single2.jpg}
}
%\begin{block}{Complete Morphological Model}
%LexModel = best path  D .o. LM \\
%where D is the dictionary transducer and LM is the language model, \\
%while .o. represents transducer composition.
%\end{block}
%One of the strengths of the WFST-based approach is the ease it provides for combining different knowledge
%sources (KSs). 
%New Operation: Search best path
%The main task for incorporating a new type of information into the overall Morphological Model is
%to convert it to WFST format %while the combination is carried out with the elegant composition algorithm

\frame
{  \frametitle{Graphics of a weighted transducer}
Example:
\begin{figure}[hp]
	\centering
		%\includegraphics[width=10cm]{C:/LatexGraphs/graph13.jpg}
	\label{fig:graph13}
\end{figure}
}


\section{Applications}

\subsection{Morphisto Analyzer/Generator at the IDS}
\frame
{  
	\frametitle{Online-Demo of the Morphisto Analyzer//Generator}
%\includegraphics[scale=0.75]{C:/LatexGraphs/Workbench.jpg} \htmladdnormallink{IDS-Webside for Morphisto}{http://www.ids-mannheim.de/ll/TextGrid/morphisto.html}\\
 \begin{itemize}
\item  \htmladdnormallink{Wordform Analysis with Morphisto}{http://ingrid.daasi.de/cgi-bin/analyze.cgi}
\item  \htmladdnormallink{Generation of Inflection Table with Morphisto}{http://ingrid.sub.uni-goettingen.de/cgi-bin/flextables.cgi} 
\end{itemize}
You are welcome to download our Morphisto Transducer Lexicon for German 
}

\subsection{Morphisto within the elexiko Project}
\frame
{
  \frametitle{Lexicographic Work in the elexiko Project}
  
\begin{itemize}
 \item Current Status: 300.000 Articles, approx. 1.000 finished
 \item Example Entry in the field \textit{Wordformation} \\
  \htmladdnormallink{elexiko-Stichwortliste}{http://www.owid.de}\\
 %\newline
%\small {
%  \itshape{Antragsberatungskommission}\\
%	\begin{alltt}
%	\hspace{0,5cm}  Wortbildungsart/-typ:	Determinativkompositum, endozentrisch \\
%	\hspace{0,5cm}  Bestandteil: \textbf{Antrag}   Wortart: Nomen\\
%	\hspace{0,5cm}  Bestandteil: \textbf{Beratungskommission}   Wortart: Nomen\\
%	\hspace{0,5cm}  Fuge: -s- \\
%	\hspace{0,5cm}  Weitere Informationen unter canoo.net}
%	\end{alltt}
\end{itemize}
$\Longrightarrow$ Generate Information Automatically
}

\subsection{Integration into the TextGrid Workbench}
\frame
{  \frametitle{Computing elsewhere}
Main Philosophy: Use web services that run on a geographically distant grid server\\
Issues
\begin{itemize} 
\item Note: The standard DeReWo-based MORPHISTO automaton has a size of 24 MB and all SFST compact format based automatons in TextGrid have a total size of ca. 150 MB
%\item Plus: Web services are platform independent
\item Challenge: Handle performance and concurrency issues on the server side\\
%Calling \textit{fst-infl2} command from the SFST toolkit vs. calling the lemmatizer daemon
\end{itemize}
For implementation details cf. Zielinski, Simon, Wittl (2009): Morphisto - Service-oriented Open Source Morphology for German, In: SFCM Z�rich, Sept. 2009
}

\frame
{  \frametitle{A Grid-based Demo}
%\includegraphics[scale=0.5] {C:/LatexGraphs/TextGrid.jpg}
%\includegraphics[width=7cm] {C:/LatexGraphs/morphoplymp1.eps}
%\includegraphics[width=7cm] {C:/LatexGraphs/configuration3.eps}

 \begin{itemize}
\item  \htmladdnormallink{TextGrid Hompage}{http://www.textgrid.de}
\end{itemize}
}




\end{document}